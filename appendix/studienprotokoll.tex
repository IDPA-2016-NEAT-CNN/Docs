\documentclass[11pt]{article}
\setlength{\parindent}{0pt}

\title{Studienprotokoll}
\author{Jan Nils Ferner, Mathias Fischler, Sara Zarubica, Jeremy Stucki}

\begin{document}
	
	\maketitle
	\newpage
	
	\tableofcontents
	\newpage
		
	\section{Verantwortlichkeiten}
	\subsection{Projektteilnehmer}
	Jan Nils Ferner \\
	Baumgartenstrasse 2 \\
	8957 Spreitenbach \\
	079 963 17 64 \\
	JanNils.Ferner@stud.bbbaden.ch\\
	
	Jeremy Stucki \\
	Oberdorf 12 \\
	5079 Zeihen \\	
	076 520 0698 \\
	Jeremy.Stucki@stud.bbbaden.ch\\
	
	Mathias Fischler \\
	Gänsacker 8 \\
	5070 Frick \\
	Mathias.Fischler@stud.bbbaden.ch\\
	
	Sara Zarubica \\
	Zelgmattstrasse 14 \\
	8956 Killwangen \\
	076 464 1306 \\
	Sara.Zarubica@stud.bbbaden.ch\\
	
	\subsection{Beteiligte Institution}
	Berufsfachschule BBB \\
	Wiesenstrasse 32 \\
	5400 Baden \\
	056 222 0206 \\
	sekretariat@bbbaden.ch\\
	\newpage
	
	\section{Das Projekt}
	\subsection{Definition des Projekts}
	In unserem Projekt wollen wir anhand von künstlicher Intelligenz Mammographien beurteilen können.
	
	Dazu verwenden wir NeuroEvolution of Augmenting Topologies (NEAT) und Convolutional Neural Network (CNN).
	
	Der Benutzer ladet eine Mammografie hoch, die er in digitaler Form (z.B. als
	.png oder .bmp) auf seinem Rechner zur Verfuegung hat. Die Software gibt
	darauf hin aus, wie wahrscheinlich die gepruefte Brust von Krebs befallen ist.
	
	Unsere Software kann Aertzte bei Zweifelsfaellen einer Diagnose unterstuetzen,
	kann diese aber keinesfalls ersetzen.
	Privatpersonen, die ueber Mammografien verfuegen, koennen die Software fuer
	praeventative Selbstuntersuchungen nutzen und je nach Ergebniss einen Artzt
	kontaktieren.
	Unsere Zielgruppe sind Aertzte, Radiologen und zum Teil Privatpersonen

	
	\subsection{Hintergrund des Projekts}
	Im Rahmen unserer Projektarbeit haben wir uns für diese Arbeit entschieden, da uns dieses Themengebiet interessiert und es in diesem Bereich noch nicht so viele Dinge gibt.

	\subsection{Der Nutzen und das Risiko}
	Unsere Software kann Aertzte bei Zweifelsfaellen einer Diagnose unterstuetzen,
	kann diese aber keinesfalls ersetzen.
	Privatpersonen, die ueber Mammografien verfuegen, koennen die Software fuer
	praeventative Selbstuntersuchungen nutzen und je nach Ergebniss einen Artzt
	kontaktieren.
	
	Zu dem Risiko würde vor allem die Ernstnahme des Resultats gehören. Damit ist gemeint dass das Resultat nicht 100 prozent ausschlaggebend ist und den Arzt nicht ersetzen kann.
	\subsection{Unsere Ziele}
	Zu unseren Zielen gehören:\\
		Das Einlesen von Mammographien\\
		Die Beurteilung von Mammographien\\
		Die Visuelle Darstellung von den generierten neuralen Netzwerken\\
		Die Plattformunabhängigkeit\\
	\subsection{Projektablauf}
	Siehe Plan
	\subsection{Projektdauer}
	Von 17.09.2016-18.01.2017
	
	
\end{document}