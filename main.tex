\documentclass[11pt]{article}

\usepackage{subfiles}
\usepackage{cite}
\usepackage[hidelinks]{hyperref}
\usepackage{ragged2e}
\usepackage{tools/neuralnetwork}
\usepackage{setspace}
\usepackage[left=2cm, right=2cm, top=2.5cm, bottom=2.5cm]{geometry}
\usepackage{graphicx}
\graphicspath{ {resources/img/} }

\hypersetup{
	allcolors=black,
	colorlinks=true
}

\setlength{\parindent}{0pt}
\setlength{\parskip}{1em}
\setstretch{1.5}

\makeatletter
\renewcommand\@biblabel[1]{#1.}
\renewcommand\@cite[1]{(#1)}
\makeatother

\title{Combining Neuro-Evolution of Augmenting Topologies with Convolutional Neural Networks}
\author{Jan Nils Ferner, Mathias Fischler, Sara Zarubica, Jeremy Stucki}

\begin{document}

\maketitle
\newpage

\begin{abstract}

	\newpage

\end{abstract}

	\tableofcontents
	\newpage

	\section{Introduction to neural networks}
		\subsection{What is a neural network?}
			\subfile{sections/"introduction to neural networks/what is a neural network"}
		\subsection{How does a neural network learn}
			\subsubsection{Traditional}
				\subfile{sections/"introduction to neural networks/how does a neural network learn/traditiontal"}
			\newpage
			\subsubsection{Genetic algorithm}
				\subfile{sections/"introduction to neural networks/how does a neural network learn/genetic algorithm"}
	\newpage

	\section{What is NEAT}
		\subfile{sections/"what is neat"}
		\subsection{Topology}
			\subfile{sections/"what is neat/topology"}
		\subsection{Speciation}
			\subfile{sections/"what is neat/speciation"}
	\newpage

	\section{Convolutional Neural Networks}
		\subsection{Problems with image recognition}
			\subfile{sections/"convolutional neural networks"/"problems with image recognition"}
		\subsection{Subsampling}
			\subfile{sections/"convolutional neural networks/subsampling/subsampling"}
			\subsubsection{Kernels}
			\subfile{sections/"convolutional neural networks/subsampling/kernerls"}
			\subsubsection{Poolers}
			\subfile{sections/"convolutional neural networks/subsampling/poolers"}
			\subsubsection{Activation function}
			\subfile{sections/"convolutional neural networks/subsampling/activation function"}
		\subsection{Structure}
			.
	\newpage

	\section{Hippocrates, a NEAT implementation}
		\subsection{Motivation}
			\subfile{sections/hippocrates/motivation}
		\subsection{Discrepancies}
			\subsubsection{Paper}
				\subfile{sections/hippocrates/discrepancies/paper}
			\subsubsection{Original implementation}
				\subfile{sections/hippocrates/discrepancies/original_implementation}
		\subsection{Visualizing Neural Networks}
			.
	\newpage

	\section{Build tools}
		\subsection{Version control}
			\subfile{sections/"build tools/version control"}
			\subsubsection{Git}
				\subfile{sections/"build tools/version control/git"}
			\subsubsection{GitHub}
				\subfile{sections/"build tools/version control/github"}
		\subsection{Integration tests}
			\subfile{sections/"integration tests"}
			\subsubsection{Travis}
				\subfile{sections/"integration tests/travis"}
			\subsubsection{AppVeyor}
				\subfile{sections/"integration tests/appveyor"}
		\subsection{CMake}
			\subfile{sections/"build tools/cmake"}
	\newpage

	\section{Combining Neuro-Evolution of Augmenting Topologies with Convolutional Neural Networks}
		\subsection{Challenges \& Solutions}
			\subfile{sections/"Combining Neuro-Evolution of Augmenting Topologies with Convolutional Neural Networks/Possibilities of modeling"}
		\subsection{Definition}
			\subfile{sections/"Combining Neuro-Evolution of Augmenting Topologies with Convolutional Neural Networks/definition"}
		\subsection{Implementation}
			.
	\newpage

	\section{Results}
		\subsection{Benchmarks}
			.
	\newpage

	\section{Further enhancements}
		\subsection{Optimisation}
			\subfile{sections/"further enhancements/optimisation"}
		\subsection{Safety concerns}
			.
		\subsection{HyperNEAT}
			\subfile{sections/"further enhancements/hyperneat"}
	\newpage

	\section{Our Work}
	We have dedicated our project work to the subject Image recognition by artificial intelligence.
	
	Image recognition by artificial intelligence has and interests us very much, because you can create something that does not exists yet. 
	Our project is mainly concerned with computer science (artificial intelligence) and medicine. 
	An important aspect for our work was to create something that can be needed in the future and what can be of benefit to other people. Because we are software engineers, it is also a good opportunity to train us in our area.
	
		\subfile{sections/collaborators}
		\subsection{Our goals}
		\subfile{sections/"our targets"}
		\subsection{Initial position}
		\subfile{sections/"initial position"}
		\subsection{Opening questions}
		\subfile{sections/"opening questions"}
		\subsection{Working programms and tools}
		\subfile{sections/"working programms and tools"}
		\subsection{Procedure}
		\subfile{sections/"procedure"}
		\subsection{Progress}
		\subfile{sections/"progress"}
		\subsection{Contact with doctors}
		\subfile{sections/"contact with doctors"}
	\newpage
	
	\nocite{*}
	\setstretch{1}

    \raggedright
    \bibliography{resources/bibliography}
	\bibliographystyle{resources/bibliographystyle}

\end{document}
