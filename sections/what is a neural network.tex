\newpage
\section{What is a neural network?}

A neural network follows the “divide and conquer” principle. It consists of two simple components: neurons and connections.

Each neuron has inputs, which are the incoming connections. It applies a simple mathematical operation to this set of inputs and returns the result.

Connections connect neurons to each other. Each connection has a weight, which determines how weak or strong the connection is.

The neurons are typically organized into layers. The first is referred to as the input layer and the last one as the output layer. The remaining layers are called hidden layers. \cite{Anderson1995}

Here is a basic example of a neural network:

{\centering
	\begin{neuralnetwork}[height=3, nodespacing=1.5cm]
		\newcommand{\nodelabel}[2]{
			\ifnum#1=0 $x_#2$ \fi
			\ifnum#1=1 $y_#2$ \fi
			\ifnum#1=2 $z_#2$ \fi
		}
		\setdefaultnodetext{\nodelabel}
		\inputlayer[count=2, bias=false, title=Input]
		\hiddenlayer[count=3, bias=false, title=Hidden] \linklayers
		\outputlayer[count=1, title=Output] \linklayers
	\end{neuralnetwork}
\par}

Each connection is represented as an arrow and has an associated weight. Every neuron is connected to all neurons in the previous and in the next layer.

The network above has 9 weights. This is the genome, which is used to store the configuration of a network.\cite{Stanley2002}

For simple networks, you can also write down the inputs and the corresponding outputs.

\[
	\begin{vmatrix} 0 & 0 \\ 0 & 1 \\ 1 & 0 \\ 1 & 1 \end{vmatrix}
	\rightarrow
	\begin{vmatrix} 0.03 \\ 0.76 \\ 0.87 \\ 0.10 \end{vmatrix}	
\]

As you can see, this network was trained to solve the XOR problem. The network outputs a higher number, if the XOR condition is met.
