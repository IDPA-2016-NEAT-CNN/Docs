A traditional approach of optimizing the connection weights to improve the network's accuracy is named backpropagation.

\begin{quote}
\emph{"The Backpropagation algorithm is a supervised learning method for multilayer feed-forward networks from the field of Artificial Neural Networks. }

\emph{Feed-forward neural networks are inspired by the information processing of one or more neural cells, called a neuron. A neuron accepts input signals via its dendrites, which pass the electrical signal down to the cell body. The axon carries the signal out to synapses, which are the connections of a cell’s axon to other cell’s dendrites."} \cite{backprop_from_scratch} 
\end{quote}

The backpropagation algorithm is a algorithm for supervised learning. In supervised learning, it is being measured how good a network performs, by testing a network with a given dataset, over and over again. 

In such a dataset, input values and the expected outputs for these values are defined.

The discrepancies from the specified outputs in the dataset and the actual outputs are called the \emph{errors} of the network.\cite{Learning2014}

Using basic calculus, the so called \emph{error} of a network can be calculated. This is also known as solving the error minimization problem. \cite{Sathyanarayana2014}

\begin{quote}
\emph{"In the most popular version of backpropagation, called stochastic backpropagation, the weights are initially set to small random values."}\cite{Sathyanarayana2014}	
\end{quote}

Stochastic methods are being used, because \emph{"properly scaled random initialization can deal with the vanishing gradient problem"}\cite{https://ai2-s2-pdfs.s3.amazonaws.com/8e56/448da09ceacea946c9d6fd393ad3e57e12cb.pdf}

With enough complexity, neural networks can represent any existing function. \cite{http://neuralnetworksanddeeplearning.com/chap4.html}

There are methods for picking initial weights, so that problems with local maximums are not limiting the backpropagation algorithm.\cite{http://ce.sharif.edu/courses/84-85/2/ce667/resources/root/Seminar_no_6/Widrow.pdf}

However, as Dr. Geoffrey E. Hinton states, backpropagation is often limited by the sheer sizes of networks that are required today:

\begin{quote}
\emph{"Backpropagation was the first computation-
	ally efficient model of how neural networks could learn
	multiple layers of representation, but it required labeled
	training data and it did not work well in deep networks."}
\cite{http://www.cs.toronto.edu/~fritz/absps/tics.pdf}
\end{quote}


https://page.mi.fu-berlin.de/rojas/neural/chapter/K7.pdf