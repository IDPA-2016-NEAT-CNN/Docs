\subsubsection{The beginning}

At the beginning of our project came the topic selection. \\
We have not dealt with this for a long time, because we already had a concrete idea. \\
To be more concrete, Jan Nils Ferner (project member) had the idea for which he could inspire us very much. \\ 
Approximately one year before the project, he had the idea to be able to recognize different images with artificial intelligence.\\ 
In the course of this year, he collected information and experiences and came up with the idea of making image recognition in the field of medicine, more precisely breast cancer (mammography). \\
So when our project started, we had a very interesting topic and also someone who had previously dealt with it.\\

\subsubsection{The planning}

At the beginning we made a rough plan of the project. \\
We have looked at the aspects which are very time-intensive and important. After this, we proceeded with the weekly planning. \\
This means we knew what we had to do now, thanks to the rough plan, but we had assembled each week to look closely at the tasks. \\
We also considered how we would like to work with the doctors and how important they are to our work.\\

\subsubsection{The realisation}

When the planning had been completed, the realization began. \\
As already mentioned in the planning, we worked in a weekly schedule. \\
This means that we divided all the big tasks into smaller ones, which go for about a week. \\
So we had a good overview of our project and knew exactly where we stand. \\
Also we had a meeting every week and discussed our concerns and problems. \\
The biggest problem with the realization was that we did not have enough computing power to run our project and visualize it. \\
So we had a problem to present our project correctly and to make it clear to the people what we have achieved with it. \\
Apart from this, the realization ran very well.\\
Our work was mainly to work on NEAT (NeuroEvolution of Augmenting Topologies) and CNN (Convolutional Neural Network). \\
When both parts were completed, it was our goal to connect them together. \\
In addition, the contact with the doctors began.  \\
On the internet and also in books we searched for doctors in the field of oncology (tumor diseases). \\
We prepared a study protocol to explain doctors our project. \\
After checking our study protocol, Dr. Serafino Forte from the hospital of Baden invited us to introduce him to our project. \\
He gave us some helpful tips and we discussed the next steps. \\

\subsubsection{The result}

Our result is a dichotomy. \\
The two largest parts, CNN (Convolutional Neural Network) and NEAT (NeuroEvolution of Augmenting Topologies), have been successfully connected together and they work. However our computing power is much too low to test our software. \\
Our problem is to explain to the people who have not dealt with our program that it would still work, even though we cannot show it to them or visualize it. \\
With a performance of a good laptop, the learning of the network would still take about a half year, which unfortunately time we not have.\\

\subsubsection{Our conclusion}

Our software works and we are very satisfied. Unfortunately, we have used an outdated technology of mammography, which is nowadays almost no longer needed. So our software can serve only in few facilities as an assistance system. However, this is not negative, even more an advantage. Inserting the new technology is possible definitely and therefore our software supports also all technologies.\\
In general, our software can be used as an assistance system, but we want to optimize the data to make it easier for the doctors to help you. \\
The project was a very good opportunity to let our ideas run free and to further train our knowledge.  \\
We are very optimistic that our software will be used with some optimizations.\\
