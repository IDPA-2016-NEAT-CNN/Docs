\subsubsection{The beginning}

Our starting point was picking a topic for the project. \\
We haven't been dealing with this for quite a long time because we already had a concrete idea of what the project was going to be about. \\
Jan Nils Ferner (project member) had an inspiring idea with which he truly inspired us. \\ 
Roughly one year before the project he had the idea to make possible the recognizing of different images using artificial intelligence.\\ 
In the course of this year, he collected information and experiences, and came up with a concept to make image recognition possible in the field of medicine, more precisely breast cancer (mammography). \\
When we started out the project, we had a very interesting topic and also someone who had previously dealt with it.\\

\subsubsection{The planning}

 At the beginning, a rough plan of how the project is going to develop was made. \\
We have looked at the aspects which are very time-consuming and important. After this, we proceeded by making a weekly planning. \\
We knew what we had to do and how to do it, of course thanks to the planning but since that wouldn't be enough, we had to group each week and discuss tasks carefully.  \\
We also discussed how we would like to cooperate with the doctors.\\

\subsubsection{The realisation}

 Since the planning has been completed, it was time for the realization. \\
As already mentioned in the planning, we worked in a weekly schedule. \\
This means that we divided all the big tasks into smaller ones, each one going on for about a week. \\
Having a really good overview of our project, we knew exactly where we stand. \\
Also we had a meeting every week and discussed our concerns about the project. \\
The biggest problem with the realization was that we lacked the computing power with which to run our project and visualize it. \\
We had a problem to present our project correctly and to make it clear to the people what we have achieved completing it. \\
Apart from this, the realization ran very well.\\
Our job was to mainly work on NEAT (NeuroEvolution of Augmenting Topologies) and CNN (Convolutional Neural Network). \\
When both parts were completed, it was our goal to connect them together. \\
In addition, we cooperated with a few doctors that helped us by providing us with data we needed.  \\
We searched for them on the internet and in books, at least the ones that were specialists in the field of oncology (tumor diseases). \\
We prepared a study protocol to present our project to the doctorsd. \\
After checking out our study protocol, Dr. Serafino Forte from the Hospital of Baden invited us to introduce him to our project. \\
He gave us some helpful tips and we discussed the next steps. \\

\subsubsection{The result}

The result led to a dichotomy. \\
The two largest parts, CNN (Convolutional Neural Network) and NEAT (NeuroEvolution of Augmenting Topologies), have been successfully connected together and they worked successfully. However, our computing power is much too low to test our software on. \\
One of our main concerns is to explain people who haven't encountered our program and this type of work that it would still run successfully, despite us not being able to show it to them or visualize it. \\
With the performance of a really good laptop, the learning of the network would still take about a half a year, which is  unfortunately time we don't have.\\

\subsubsection{Our conclusion}

 Our software worked successfully and we are very satisfied with it. Unfortunately, we have used an outdated technology of mammography, which is nowadays almost to no longer needed. So it turns out that our software can serve in only a few facilities as an assistance system. However, this doesn't necessarily mean it's negative, quite the contrary - an advantage. Inserting the new technology is definitely possible, and therefore our software would also support all technologies.\\
In conclusion, our software can be used as an assistance system, but we want to optimize the data to make it easier for the doctors to work with it and eventually be able to help people thanks to the provided information. \\
The project was a very good opportunity to let our ideas run free and to further train our knowledge.  \\
We are very optimistic for our software to be used and in future optimized.\\
