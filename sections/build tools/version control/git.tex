Git is a version control system that was first proposed by Linus Torvalds in 2005. \cite{Torvalds} \\
It is a free, open-source version control system, which we use for our entire source code and documentation.

We first separated our code into multiple repositories \cite{gitrepo} as we thought it would make sense to keep NEAT and CNN separated. Later we decided that it would make more sense to keep everything in a single repository, as we had to use both parts simultaneously.

A repository is like a project folder, but it is synced across multiple computers.

Git has many powerful tools that defeat their antecessors from other version control by a big margin in terms of usability, performance and stability.

One of these tools is the merge tool. It allows to either automatically - if no conflicts happen - or manually - merge together files from different branches or repositories. This is very useful when working together with multiple teammembers, because you don't have to watch out too much about working in the same files - as long as there's no redundant work done - because the merge tool is able to often fix alot of collisions automatically, or if not, it marks the colliding parts so users have less hard times fixing the conflicts.

When creating and pushing commits onto git repository (a commit is a subset of changes) everyone gets a copy of this commit, as soon as queried for it via "pulling" (getting the latest changes from a remote repository.)\cite{Torvalds}

Because of that commit messages are important. They ought to explain what the commit changed on the repository.

To make sure everyone can understand what has been done, we adopted some rule set for naming commits\cite{Beams2014}:

\begin{itemize}
	\item Separate subject from body with a blank line
	\item Limit the subject line to 50 characters
	\item Capitalize the subject line
	\item Do not end the subject line with a period
	\item Use the imperative mood in the subject line
	\item Wrap the body at 72 characters
	\item Use the body to explain what and why vs. how
\end{itemize}

The \emph{limit the subject line to 50 characters} rule is very useful:

It guarantees that on github, the commit message will be readable without requiring a user to expand a area of the page.

The \emph{use the imperative mood in the subject line} rule is useful because it makes commits more readable. As we have used this rule, it has become more and more clear to us that not using imperative means having redundant characters.

As an example, instead of "Add Implementation" the commit message could be "Added Implementation". That is two characters more without any gain of insight or readability. Thats why we found this rule particularly useful.
